\documentclass{article}
\usepackage{multicol}
\usepackage[scale=.85]{geometry}
\usepackage{siunitx}
\usepackage{fontspec}
\setmainfont{TeX Gyre Adventor}
\usepackage{tikz}
\usetikzlibrary{tikzmark,shapes.geometric}
\title{Constructing a Christmas Tree}
\thispagestyle{empty}
\pagestyle{empty}

\newcounter{tree}

\makeatletter
\tikzset{
  fill colour/.code={%
    \tikz@addoption{\pgfsetfillcolor{#1}}%
    \def\tikz@fillcolor{#1}%
  },
  draw colour/.code={%
    \tikz@addoption{\pgfsetstrokecolor{#1}}%
    \def\tikz@strokecolor{#1}%
  },
}
\makeatother
\tikzset{
  dot/.style={
    minimum width=2.5mm,
    ellipse,
    fill,
    node contents={},
    on step=8{fill=none}
  },
  construction line/.style={
    draw=gray
  },
  new line/.style={
    draw=black
  },
  on step/.code 2 args={
    \ifnum#1=\the\value{tree}\relax
    \pgfkeysalso{#2}%
    \fi
  },
  before step/.code 2 args={
    \ifnum#1>\the\value{tree}\relax
    \pgfkeysalso{#2}%
    \fi
  },
  after step/.code 2 args={
    \ifnum#1<\the\value{tree}\relax
    \pgfkeysalso{#2}%
    \fi
  },
  from step/.style 2 args={
    on step={#1}{#2},%
    after step={#1}{#2}%
  },
  between steps/.code args={#1 and #2#3}{
    \ifnum#1<\the\value{tree}\relax
    \ifnum#2>\the\value{tree}\relax
    \pgfkeysalso{#3}%
    \fi
    \fi
    \ifnum#1=\the\value{tree}\relax
    \pgfkeysalso{#3}%
    \fi
    \ifnum#2=\the\value{tree}\relax
    \pgfkeysalso{#3}%
    \fi
  },
  colour step/.style={
    before step={#1}{fill=none,draw=none},
    on step={#1}{fill colour=black,draw colour=black},
    after step={#1}{fill colour=gray,draw colour=gray},
  },
  colour steps between/.style args={#1 and #2}{
    before step={#1}{fill=none,draw=none},
    between steps=#1 and #2{fill colour=black,draw colour=black},
    after step={#2}{fill colour=gray,draw colour=gray},
  },
  pics/christmas tree/.style={
    code={
      \stepcounter{tree}
      \draw (0,0) rectangle (21,29);
      \clip (0,0) rectangle (21,29);
      \path (7,24) coordinate (a);
      \path (14,24) coordinate (b);
      \path (10.5,24) coordinate (c);
      \path (c) ++(-120:3.5) coordinate (d);
      \path (c) ++(-60:3.5) coordinate (e);
      \path (c) ++(-150:3.5) coordinate (f);
      \path (c) ++(-30:3.5) coordinate (g);
      \path (a) ++(-150:3.5) coordinate (i);
      \path (b) ++(-30:3.5) coordinate (j);
      \path (f -| c) coordinate (h);
      \draw[colour step=1] (1,24) -- (20,24);
      \draw[colour step=2] (10.5,28) -- (10.5,10);
      \draw[colour step=3] (10.5,24) arc[radius=3.5,start angle=0,delta angle=-180];
      \draw[colour step=3] (10.5,24) arc[radius=3.5,start angle=-180,delta angle=180];
      \draw[colour step=3] (14,24) arc[radius=3.5,start angle=0,delta angle=-180];
      \draw[colour step=4] (c) -- (d);
      \draw[colour step=4] (c) -- (e);
      \draw[colour step=4] (10.5,24) -- ++(-150:8);
      \draw[colour step=4] (10.5,24) -- ++(-30:8);
      \draw[colour step=5] ([xshift=-1cm]i) -- ([xshift=1cm]j);
      \draw[colour step=6] (i) arc[radius=6.531,start angle=-180,end angle=0];
      \draw[colour step=6] (h) arc[radius=6.531,start angle=-180,end angle=0];
      \draw[colour step=6] (h) arc[radius=6.531,start angle=0,end angle=-180];
      \draw[colour steps between=7 and 8,on step=8{thick}] (10.5,24) arc[radius=3.5,start angle=0, end angle=-60] arc[radius=3.5, start angle=-120, end angle=-60] arc[radius=3.5, start angle=-120, end angle=-180];
      \draw[colour steps between=7 and 8,on step=8{thick}] (h) arc[radius=6.531,start angle=0, end angle=-60] arc[radius=6.531, start angle=-120, end angle=-60] arc[radius=6.531, start angle=-120, end angle=-180];
      \node[at=(a),dot,colour step=1,between steps=1 and 7{label=90:{\(A\)}}];
      \node[at=(b),dot,colour step=1,between steps=1 and 7{label=90:{\(B\)}}];
      \node[at=(c),dot,colour step=2,between steps=2 and 7{label=45:{\(C\)}}];
      \node[at=(d),dot,colour step=3,between steps=3 and 7{label=-90:{\(D\)}}];
      \node[at=(e),dot,colour step=3,between steps=3 and 7{label=-90:{\(E\)}}];
      \node[at=(f),dot,colour step=4,between steps=4 and 7{label=-90:{\(F\)}}];
      \node[at=(g),dot,colour step=4,between steps=4 and 7{label=-90:{\(G\)}}];
      \node[at=(h),dot,colour step=5,between steps=5 and 7{label=-90:{\(H\)}}];
      \node[at=(i),dot,colour step=5,between steps=5 and 7{label=-90:{\(I\)}}];
      \node[at=(j),dot,colour step=5,between steps=5 and 7{label=-90:{\(J\)}}];
    }
  }
}

\begin{document}

\section*{Constructing a Christmas Tree}

Draw all construction lines and labels lightly.

\begin{enumerate}
\item
\begin{multicols}{2}
Begin with a horizontal line across your page about \SI{5}{cm} from the top.
Mark points on your line about \SI{7}{cm} in from each edge of the paper.
Label these points \(A\) and \(B\).

\columnbreak
\begin{tikzpicture}[scale=.2]
\pic[transform shape]{christmas tree};
\end{tikzpicture}
\end{multicols}

\item
\begin{multicols}{2}
Construct the perpendicular bisector of \(A\) and \(B\).
Label the point where this crosses the original line as \(C\).

\columnbreak
\begin{tikzpicture}[scale=.2]
\pic[transform shape]{christmas tree};
\end{tikzpicture}
\end{multicols}

\item
\begin{multicols}{2}
Set your compass to the distance between \(A\) and \(C\).
Draw three semi-circles centred at \(A\), \(B\), and \(C\) going below the original line.
Label the points where the middle semi-circle crosses each of the other two as \(D\) and \(E\).

\columnbreak
\begin{tikzpicture}[scale=.2]
\pic[transform shape]{christmas tree};
\end{tikzpicture}
\end{multicols}

\newpage
\item 
\begin{multicols}{2}
Draw a line from \(C\) to \(D\) and from \(C\) to \(E\).
Bisect angle \(ACD\) and then bisect angle \(BCE\).
Make your lines long enough to intersect the middle semi-circle.
Label the points where the lines meet that semi-circle \(F\) and \(G\).

\columnbreak
\begin{tikzpicture}[scale=.2]
\pic[transform shape]{christmas tree};
\end{tikzpicture}
\end{multicols}

\item
\begin{multicols}{2}
Draw the line joining \(F\) and \(G\) and extend it to meet the two outer semi-circles.
Mark where this line crosses the vertical line and label it \(H\).
Mark where this line crosses the two outer semi-circles \(I\) and \(J\).

\columnbreak
\begin{tikzpicture}[scale=.2]
\pic[transform shape]{christmas tree};
\end{tikzpicture}
\end{multicols}

\item
\begin{multicols}{2}
Set your compasses to the distance between \(H\) and \(F\).
Draw semi-circles at \(F\), \(G\), and \(H\) (as best fits on the paper).

\columnbreak
\begin{tikzpicture}[scale=.2]
\pic[transform shape]{christmas tree};
\end{tikzpicture}
\end{multicols}

\item
\begin{multicols}{2}
Now draw the following arcs in a heavier style.
Start with the first three semi-circles that you drew (centred at \(A\), \(C\), and \(B\)).
Starting at \(C\), follow the left-hand semi-circle (centred at \(A\)) until it meets the middle semi-circle.
Then follow the middle semi-circle to the right until it meets the right-hand semi-circle.
Finally, follow that back to \(C\).

Repeat this for the lower three semi-circles (you may wish to not draw the part near the point \(H\))

\columnbreak
\begin{tikzpicture}[scale=.2]
\pic[transform shape]{christmas tree};
\end{tikzpicture}
\begin{tikzpicture}[scale=.2]
\pic[transform shape]{christmas tree};
\end{tikzpicture}
\end{multicols}

\end{enumerate}
\end{document}
